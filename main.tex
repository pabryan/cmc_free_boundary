\documentclass{amsart}

%\usepackage{etoolbox}
%\makeatletter
%\let\ams@starttoc\@starttoc
%\makeatother
%\makeatletter
%\let\@starttoc\ams@starttoc
%\patchcmd{\@starttoc}{\makeatletter}{\makeatletter\parskip\z@}{}{}
%\makeatother

%\usepackage[parfill]{parskip}

\usepackage[colorlinks=true,linkcolor=blue,citecolor=blue,urlcolor=blue]{hyperref}
\usepackage{bookmark}
\usepackage{amsthm,thmtools,amssymb,amsmath,amscd}

\usepackage[bibstyle=alphabetic,citestyle=alphabetic,backend=bibtex]{biblatex}
\bibliography{Bibliography}

\usepackage{fancyhdr}
\usepackage{esint}

\usepackage{enumitem}

\usepackage{pictexwd,dcpic}

\usepackage{graphicx}

\declaretheorem[name=Theorem,numberwithin=section]{thm}
\declaretheorem[name=Remark,style=remark,sibling=thm]{rem}
\declaretheorem[name=Lemma,sibling=thm]{lemma}
\declaretheorem[name=Proposition,sibling=thm]{prop}
\declaretheorem[name=Definition,style=definition,sibling=thm]{defn}
\declaretheorem[name=Corollary,sibling=thm]{cor}
\declaretheorem[name=Assumption,style=remark,sibling=thm]{ass}
\declaretheorem[name=Example,style=remark,sibling=thm]{example}


\numberwithin{equation}{section}

\usepackage{cleveref}
\crefname{lemma}{Lemma}{Lemmata}
\crefname{prop}{Proposition}{Propositions}
\crefname{thm}{Theorem}{Theorems}
\crefname{cor}{Corollary}{Corollaries}
\crefname{defn}{Definition}{Definitions}
\crefname{example}{Example}{Examples}
\crefname{rem}{Remark}{Remarks}
\crefname{ass}{Assumption}{Assumptions}
\crefname{not}{Notation}{Notation}

%Symbols
\renewcommand{\~}{\tilde}
\renewcommand{\-}{\bar}
\newcommand{\bs}{\backslash}
\newcommand{\cn}{\colon}
\newcommand{\sub}{\subset}

\newcommand{\N}{\mathbb{N}}
\newcommand{\R}{\mathbb{R}}
\newcommand{\Z}{\mathbb{Z}}
\renewcommand{\S}{\mathbb{S}}
\renewcommand{\H}{\mathbb{H}}
\newcommand{\C}{\mathbb{C}}
\newcommand{\K}{\mathbb{K}}
\newcommand{\Di}{\mathbb{D}}
\newcommand{\B}{\mathbb{B}}
\newcommand{\8}{\infty}

%Greek letters
\renewcommand{\a}{\alpha}
\renewcommand{\b}{\beta}
\newcommand{\g}{\gamma}
\renewcommand{\d}{\delta}
\newcommand{\e}{\epsilon}
\renewcommand{\k}{\kappa}
\renewcommand{\l}{\lambda}
\renewcommand{\o}{\omega}
\renewcommand{\t}{\theta}
\newcommand{\s}{\sigma}
\newcommand{\p}{\varphi}
\newcommand{\z}{\zeta}
\newcommand{\vt}{\vartheta}
\renewcommand{\O}{\Omega}
\newcommand{\D}{\Delta}
\newcommand{\G}{\Gamma}
\newcommand{\T}{\Theta}
\renewcommand{\L}{\Lambda}

%Mathcal Letters
\newcommand{\cL}{\mathcal{L}}
\newcommand{\cT}{\mathcal{T}}
\newcommand{\cA}{\mathcal{A}}
\newcommand{\cW}{\mathcal{W}}

%Mathematical operators
\newcommand{\INT}{\int_{\O}}
\newcommand{\DINT}{\int_{\d\O}}
\newcommand{\Int}{\int_{-\infty}^{\infty}}
\newcommand{\del}{\partial}

\newcommand{\inpr}[2]{\left\langle #1,#2 \right\rangle}
\newcommand{\fr}[2]{\frac{#1}{#2}}
\newcommand{\x}{\times}
\DeclareMathOperator{\Tr}{Tr}

\DeclareMathOperator{\dive}{div}
\DeclareMathOperator{\id}{id}
\DeclareMathOperator{\Id}{Id}
\DeclareMathOperator{\pr}{pr}
\DeclareMathOperator{\Diff}{Diff}
\DeclareMathOperator{\supp}{supp}
\DeclareMathOperator{\graph}{graph}
\DeclareMathOperator{\osc}{osc}
\DeclareMathOperator{\const}{const}
\DeclareMathOperator{\dist}{dist}
\DeclareMathOperator{\loc}{loc}
\DeclareMathOperator{\grad}{grad}
\DeclareMathOperator{\ric}{Ric}
\DeclareMathOperator{\Rm}{Rm}
\DeclareMathOperator{\weingarten}{\mathcal{W}}
\DeclareMathOperator{\inj}{inj}
\DeclareMathOperator{\SO}{SO}

%Environments
\newcommand{\Theo}[3]{\begin{#1}\label{#2} #3 \end{#1}}
\newcommand{\pf}[1]{\begin{proof} #1 \end{proof}}
\newcommand{\eq}[1]{\begin{equation}\begin{alignedat}{2} #1 \end{alignedat}\end{equation}}
\newcommand{\IntEq}[4]{#1&#2#3	 &\quad &\text{in}~#4,}
\newcommand{\BEq}[4]{#1&#2#3	 &\quad &\text{on}~#4}
\newcommand{\br}[1]{\left(#1\right)}

\newcommand{\abs}[1]{\left|{#1}\right|}


%Logical symbols
\newcommand{\Ra}{\Rightarrow}
\newcommand{\ra}{\rightarrow}
\newcommand{\hra}{\hookrightarrow}
\newcommand{\mt}{\mapsto}

%Names
\newcommand{\schrodinger}{Schr\"{o}dinger}

%Notes
\newcommand{\pa}[1]{{\color{green} pa: {#1}}}
\newcommand{\jj}[1]{{\color{red} jj: {#1}}}
\newcommand{\mni}[1]{{\color{blue} mni: {#1}}}

%Fonts
\newcommand{\mc}{\mathcal}
\renewcommand{\it}{\textit}
\newcommand{\mrm}{\mathrm}

%Spacing
\newcommand{\hp}{\hphantom}


%\parindent 0 pt

\protected\def\ignorethis#1\endignorethis{}
\let\endignorethis\relax
\def\TOCstop{\addtocontents{toc}{\ignorethis}}
\def\TOCstart{\addtocontents{toc}{\endignorethis}}

\DeclareMathOperator{\nor}{N}
\DeclareMathOperator{\norS}{N_{\S^n}}


\title[]{CMC surfaces with free boundary on the sphere}

\curraddr{}
\email{}

\dedicatory{}
\subjclass[2010]{}
\keywords{}

\begin{document}

\begin{abstract}
\end{abstract}

\maketitle

\section{The set up}

Let \(M^n \subseteq \R^{n+1}\) be a smooth, oriented \(n\)-dimensional hypersurface with boundary \(\partial M\) lying on the unit sphere \(\S^n = \{\|x\| = 1\}\). Let \(\nor\) denote the unit normal to \(M\), \(H\) the mean curvature of \(M\) with respect to \(\nor\) and \(\norS\) the unit normal to the sphere \(\S^n\). Also let \(s = \inpr{X}{\nor}\) denote the support function of \(M\).

We assume that,
\begin{enumerate}
\item \(H \equiv \text{const}\) (CMC),
\item \(\inpr{\nor}{\norS} = 0\) along \(\partial M\) (\(M\) meets \(\S^n\) orthogonally),
\item \(s = \inpr{\nor}{X} > 0\) where \(X(x) = x\) is the position vector field on \(\R^{n+1}\) (\(M\) is star-shaped).
\end{enumerate}  
Note that \(\norS = X\) along \(\S^n\) and so we also have
\begin{enumerate}[resume]
\item \(s = \inpr{\nor}{X} = \inpr{\nor}{\norS} = 0 \text{ along } \partial M\) (zero boundary condition).
\end{enumerate}
The orthogonalality condition \(\inpr{\nor}{\norS} = 0\) implies that
\begin{enumerate}[resume]
\item \(\norS\) is tangent to \(M\) along \(\partial M\).
\end{enumerate}
Furthermore, since \(\partial M \subseteq \S^n\), we have \(T\partial M \subseteq TS^n\) along \(\partial M\) and hence \(T\partial M \perp N_{\S^n}\). We also have \(T\partial M \subseteq TM\) along \(\partial M\) and so \(T\partial M \perp N\). These two facts combined imply that
\begin{enumerate}[resume]
\item \(N_{\partial M} = N_{\S^n}\) along \(\partial M\).
\end{enumerate}

\section{Conventions}

We define the second fundamental form, \(h\) by
\[
D_X Y = \nabla_X Y - h(X, Y) N
\]
or in other words,
\[
h(X, Y) = -\inpr{D_X Y}{N} = \inpr{Y}{D_X N}.
\]

\section{The computation}

We use the following:
\begin{align}
\Delta |X|^2 &= -2 H s + 2n, \label{eq:DeltaX2} \\
\Delta s &= H - |A|^2 s. \label{eq:Deltas}
\end{align}

Integrating equation \eqref{eq:DeltaX2} gives
\[
\int_{\partial M} \inpr{X}{\nabla_{\norS} X} = \frac{1}{2} \int_M \Delta |X|^2 = -H \int_M s + n |M|.
\]
Observe that \(\nabla X = \Id\) and as noted, along \(\partial M\), \(X = \nor\) hence \(\inpr{X}{\nabla_{\norS} X} = \inpr{\norS}{\norS} = 1\). Therefore, \(|\partial M| = -H \int_M s + n |M|\) which we write as
\begin{equation}
\label{eq:intDeltaX2}
n |M| = H \int_M s + |\partial M|.
\end{equation}

Integrating equation \eqref{eq:Deltas} gives
\[
\begin{split}
H |M| - \int_M |A|^2 s &= \int \Delta s = \int_{\partial M} \nabla_{\norS} s \\
&= \int_{\partial M} \nabla_{\norS} \inpr{X}{\nor} = \int_{\partial_M} \inpr{\norS}{\nor} + \inpr{X}{\nabla_{\norS} \nor} \\
&= \int_{\partial_M} A(\norS, \norS).
\end{split}
\]
That is,
\begin{equation}
\label{eq:intDeltas}
H |M| = \int_M |A|^2 s + \int_{\partial M} A(\norS, \norS).
\end{equation}

Then from equations \eqref{eq:intDeltaX2} and \eqref{eq:intDeltas} we deduce
\[
H^2 \int_M s + H |\partial M| = n H |M| = n \int_M |A|^2 s + n \int_{\partial M} A(\norS, \norS).
\]
That is
\[
\int_{\partial_M} H - n A(\norS, \norS) = \int_M (n|A|^2 - H^2) s
\]
Notice that the term \(n|A|^2 - H^2 \geq 0\) equalling \(0\) precisely at umbilic points and that the star shaped assumption is \(s > 0\).

\end{document}
