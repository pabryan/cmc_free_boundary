\documentclass[10pt]{amsart}

\usepackage{amssymb, amsthm, amsmath, graphicx, yfonts, wasysym, appendix, url, verbatim}
\usepackage[margin=4cm]{geometry}
\usepackage{rotating}
\usepackage{amsbsy,enumerate}
\usepackage{graphicx}
\usepackage{ccaption}
\usepackage{xcolor}
\usepackage{times}
\usepackage{wrapfig}
\usepackage{amsmath, amssymb, amsthm} %Also recommend the standard AMS LaTeX maths packages.

%%%%%%%%%%%%%%%%%%%%%%%%%%%%%%%%%%%%%%%%%
%%  Some useful macros.
\newcommand{\pD}[2]{\frac{\partial #1}{\partial #2}}
\newcommand{\IT}[1]{\intertext{\flushright\small\parbox{3in}{\flushright #1}}}
\newcommand{\HL}[1]{\textbf{\textit{#1}}}
\newcommand{\auth}[1]{\hskip+1in\textit{Authored by #1}.\vspace{1cm}}
\newcommand{\T}[1]{\tilde{#1}}
\newcommand{\myover}[2]{\genfrac{}{}{0pt}{}{#1}{#2}}
\newcommand{\vn}[1]{\lVert#1\rVert}
\newcommand{\IP}[2]{\left< #1 , #2 \right>}
\newcommand{\eIP}[2]{\left(\left. #1 \hskip+0.5mm\right| #2 \right)}
\newcommand{\rD}[2]{\frac{d #1}{d #2}}
\newcommand{\tpD}[2]{\text{\tiny$\frac{\partial #1}{\partial #2}$}}
\newcommand{\sfrac}[2]{\text{\fontsize{5}{5}\selectfont$\frac{#1}{#2}$}}

%%%%%%%%%%%%%%%%%%%%%%%%%%%%%%%%%%%%%%%%%
%%  Some non-LaTeX native symbols I use.
\newcommand{\arccosh}{\text{arccosh}}
\newcommand{\Q}{\ensuremath{\mathcal{Q{}}}}
\renewcommand{\P}{\ensuremath{\mathcal{P{}}}}
\newcommand{\R}{\ensuremath{\mathbb{R}}}
\newcommand{\D}{\ensuremath{\mathbb{D}}}
\newcommand{\N}{\ensuremath{\mathbb{N}}}
\renewcommand{\S}{\ensuremath{\mathbb{S}}}
\newcommand{\Z}{\ensuremath{\mathbb{Z}}}
\newcommand{\SD}{\ensuremath{\mathcal{D{}}}}
\newcommand{\SB}{\ensuremath{\mathcal{B{}}}}
\newcommand{\SW}{\ensuremath{\mathcal{W{}}}}
\newcommand{\SY}{\ensuremath{\mathcal{Y{}}}}
\newcommand{\SH}{\ensuremath{\mathcal{H{}}}}
\newcommand{\SK}{\ensuremath{\mathcal{K{}}}}
\newcommand{\SL}{\ensuremath{\mathcal{L{}}}}
\newcommand{\SA}{\ensuremath{\mathcal{A{}}}}
\newcommand{\SX}{\ensuremath{\mathcal{X{}}}}
\newcommand{\SI}{\ensuremath{\mathcal{I{}}}}
\newcommand{\SP}{\ensuremath{\mathcal{P{}}}}
\newcommand{\SU}{\ensuremath{\mathcal{U{}}}}
\newcommand{\SC}{\ensuremath{\mathcal{C{}}}}
\newcommand{\SG}{\ensuremath{\mathcal{G{}}}}
\newcommand{\SR}{\ensuremath{\mathcal{R{}}}}
\renewcommand{\SS}{\ensuremath{\mathcal{S{}}}}

\newcommand{\Div}{\text{div}}
\renewcommand{\L}{\ensuremath{\mathcal{L}}}
\newcommand{\g}{\textswab{g}}
\newcommand{\Cmss}{\ensuremath{C_{M\!S\!S}}}
\newcommand{\ts}[2]{\ensuremath{\text{$\textstyle\frac{#1}{#2}$}}}
\newcommand{\lb}{\ensuremath{<\hskip-2mm<}}
\newcommand{\rb}{\ensuremath{>\hskip-2mm>}}
\newcommand{\gmcfh}{GMCF${}^{H \le\, 0}$}

\allowdisplaybreaks[4]

%%%%%%%%%%%%%%%%%%%%%%%%%%%%%%%%%%%%%%%%%
%%  Theorem environments.
\newtheorem{thm}{Theorem}[section]
\newtheorem{cor}[thm]{Corollary}
\newtheorem{prop}[thm]{Proposition}
\newtheorem{lem}[thm]{Lemma}
\newtheorem{id}[thm]{Identity}
\newtheorem{est}[thm]{Estimate}
\newtheorem*{uthm}{Theorem}
\newtheorem*{uprop}{Proposition}
\newtheorem*{ulem}{Lemma}
\newtheorem*{defn}{Definition}
\newtheorem{conds}[thm]{Conditions}
\newtheorem{ass}[thm]{Assumption}

\theoremstyle{remark}
\newtheorem*{rmk}{Remark}
\newtheorem*{note}{Note}
\newtheorem*{cond}{Condition}

\title{CMC in the Ball yo}
\author{}
\keywords{minimal hypersurface, mean curvature, free boundary, geometric
analysis} \subjclass[2000]{49Q05\and 53A10}


\begin{document}

\begin{abstract}
In this article we study CMC hypersurfaces with free boundary.
We use the strong maximum principle to show that CMC hypersurfaces with free
boundary in (for the outside we need a noncompact max principle working for
free boundary hypersurfaces in the ball) the ball and graphical in a rotation
of the Euclidean space are either flat or rotationally symmetric.  This is
independent of topology or dimension.
We also give a classification of free boundary CMC surfaces without an
additional hypothesis on their stability.
\end{abstract}

\maketitle

\section{Introduction}%{{{


\begin{thm}[Graphical in a rotation CMC are either flat or rotationally symmetric]
Let $F:M^n\rightarrow\R^{n+1}$ be a smooth rotation Killing-graphical CMC
hypersurface with free boundary on $\S^n\subset\R^{n+1}$. Let the graph direction $K$ be a rotation of the Euclidean space. Then $M^n = \D^n$
and $M := F(M^n)$ is a standard flat disk or it is rotationally symmetric with rotation $K$.
\label{thmuniqueness}
\end{thm}

In the above statement, we use Killing-graphical to mean that the function
$s:M\rightarrow\R$ given by
\begin{equation}
\label{EQsV}
s(x) := \IP{\nu^M(x)}{V(x)}\,,
\end{equation}
where $\nu^M:M\rightarrow\R^{n+1}$ is a unit normal vector field along $F(M^n)$,
and $V:\R^{n+1}\rightarrow\R^{n+1}$ is a Killing field,
is positive.

\rmk{Weakly graphical only} We do not require the strict graphicality. If that is true then we obtain uniqueness of CMC disks.

%}}}

\section{Setting}%{{{

Consider the standard unit sphere in Euclidean space $\S^n = \{x\in\R^{n+1}\,:\,|x|=1\}\subset
\R^{n+1}$.
We use $\nu^{\S^n}:\S^{n}\rightarrow\R^{n+1}$ to denote its outer
normal vectorfield.
Let $M^n$ be a smooth, orientable $n$-dimensional Hausdorff paracompact manifold with boundary $\partial M^n$.
Let $g$ be a Riemannian metric on $M^n$.
Set $M:=F(M^n)\subset{\R}^{n+1}$ where $F:M^n\rightarrow {\R}^{n+1}$ is a
smooth isometric immersion satisfying
\begin{align}
&{\partial} M \ \equiv \ F({\partial} M^n)\ =\ M \cap \S^n,\notag\\
&\IP{{\nu}^M}{{\nu}^{\S^n}}(F(p))\ =\ 0, \text{ }\forall
p~\in{\partial} M^n.
\label{initial}
\end{align}
Since $F$ is isometric, the Riemannian structure induced by the embedding $F$
is the same as that given by $g$, that is $(M^n, g) = (M^n, F^*\delta)$, where
$\delta$ is the standard metric on $\R^n$.

Let us denote by $A:T M\times T M \rightarrow \R$ the second fundamental form
of $M$ with components given by $h_{ij}$ where $1\leq i,j\leq n$ where $h_{ij}=
A(\tau_i,\tau_j)$ for two sections $\tau_i$ and $\tau_j$ in $TM$.
For $\S^n$ we have $A^{\S^n}:T \S^n \times T\S^n\rightarrow \R$ the
second fundamental form with components $h^{\S^n}_{ij}$ for $1\leq i,j\leq
n$.
%}}}

\section{Auxiliary Equations}%{{{

Let us define the quantities we use in the proof of the uniqueness theorem.
Recall the function $s:M\rightarrow\R$ defined in \eqref{EQsV} above.
When $V$ is a translation, following \cite{ecker1989mce} we term $s$ the \emph{graph quantity}.
If, up to reparametrisation,
\[
F(p) = (p,u(p)) = p_ie_i + u(p)V\,,
\]
then one can relate the gradient of the associated scalar function $u$ to the
reciprocal of the graph quantity $s$. This implies that a lower bound on $s$
is equivalent to a gradient bound for $u$.
Throughout this section we assume that $F:M^n\rightarrow\R^{n+1}$ is a smooth
minimal immersed hypersurface.

\begin{lem} The quantity $s:M\rightarrow\R$
satisfies
\begin{align*}
{\Delta}^M s = - |A|^2 s\,.
\end{align*}
where $\Delta^M$ is the Laplace-Beltrami operator on $M$.
\label{sevolution}
\end{lem}

Note that for any choice of orthonormal basis of the tangent space of $\S^n$
the second fundamental form of $\S^n$ is diagonal.
It was shown by Stahl \cite{stahl1996convergence} that on the boundary we have
\begin{align*}
h_{in}=h^{\S^n}_{in}=0\quad \text{ for all }\quad i=1,\ldots,n-1.
\end{align*}
Since $n-1$ of the tangent vectors are on the closed submanifold $\partial M\subset\S^n$ we can choose
our basis above such that on the boundary
\begin{align*}
h_{ij}=0\quad  \text{ if } \quad i\neq j\,,\quad \text{ and }\quad h^{\S^n}_{ij}=\delta_{ij}\quad\text{ if }\quad i\neq j.
\end{align*}
\section{Minimal graphical hypersurfaces with free boundary are flat disks}
\label{KillingSection}

Now assume that
\begin{align}
	s \geq 0 \label{graphcondition}\,.
\end{align}
We shall prove Theorem \ref{thmuniqueness}.

\begin{proof}
Apply the elliptic maximum principle to the evolution of $s0\geq 0$ to see that
there is no interior minima. On the boundary we note that $\nabla_{\nu_\S^n}s=0$ using
$=\nu^{\S^n}$, the Neumann boundary condition and the  of the second fundamental form on the boundary. 

\begin{align*}
\nabla_{\nu_\S^n}s_v = h_{nn}\IP{\nu^{\S^n}}{K} = 0\,.
\end{align*}
This implies that $s_v$ has no boundary minima either.  Therefore $s_v$ is constant everywhere on $\bar M$.

Returning to the evolution of $s_v$ on the interior this tells us that
\begin{align*}
{\Delta}^M s_V = - |A|^2 s_V\,,
\end{align*}
implying that either $M$ is a flat disc or $K\in TM$ everywhere on $M$, making $M$ rotationally symmetric. 
\end{proof}

\begin{rmk}
The bounded curvature is required for the evolution of the $s_v$.
\end{rmk}

%}}}

\section{Starshaped implies umbilic}%{{{

A theorem of Jellet says that the only starshaped CMC surface in $\R^3$ is the round sphere.
We wish to redo this for hypersurfaces with free boundary on the sphere.
Right now we have something that works for surfaces.

First some lemmata.

\begin{lem}
Suppose $\gamma:\S^1\rightarrow\R^3$ is a smooth closed curve such that $|\gamma| = 1$ (that is, lies on a unit sphere).
Then the total torsion of $\gamma$ is zero.
\end{lem}
\begin{proof}
Exercise yo.
\end{proof}

\begin{lem}
Suppose $F:M^2\rightarrow\R^3$ is a smooth starshaped CMC surface with free boundary on
$\S^2\subset\R^{3}$.
Parametrise a connected component of the boundary $\partial M^i = \S^1$ by a smooth immersed curve $\gamma:\S^1\rightarrow\R^3$.
Take Fermi coordinates in an annulus around $\partial M^i$, with frame oriented
such that the first vector $T_1$ equals $\partial_s\gamma$ (here $\partial_s$
is the arclength derivative along $\gamma$).
Then
\[
	\int_{\partial M^i} h_{11}\,dS = \text{sumthin}\,.
\]
\end{lem}
\begin{proof}
Calculating, $\gamma \perp \gamma_s$ and
\[
	0 = \partial^2_s|\gamma|^2 = \IP{\gamma}{\gamma_{ss}} + 1
	\,,
\]
so the curvature vector for $\gamma$ has unit length projection onto the normal bundle of the sphere.
Now the Frenet-Serret formula tells us that (here we use the notation that
$\{T,N,B\}$ denote the tangent, normal, and binormal respectively)
\[
	T_s = \kappa N\,,\qquad
	N_s = -\kappa T + \tau B\,,\qquad
	B_s = -\tau N\,.
\]
We wish to compare this frame with $\{\gamma_s,\gamma,\gamma_s\times\gamma\}$.
(Note that this frame is $\{T,\nu^{\S^2},\nu\}$.)
Clearly we have
\[
	\gamma\times (\gamma_s\times\gamma) = \gamma_s\,,\qquad
	(\gamma_s\times\gamma)\times\gamma_s = \gamma\,.
\]
Note that $\gamma_s = T$.
Since $h_{11} = \IP{\gamma_{ss}}{\gamma_s\times\gamma}$ and $\gamma_{ss} \perp \gamma_s$, we find
\[
	\gamma_{ss} = -\gamma + \IP{\gamma_s\times\gamma}{\gamma_{ss}}\gamma_s\times\gamma
	 = -\gamma + h_{11}\nu
	\,.
\]
We verify that
\[
	\int_{\S} h_{11}\,ds
	= \int_{\S} \IP{\gamma_{ss}}{\gamma_s\times\gamma}\,ds
	= - \int_{\S} \IP{\gamma_{s}}{\gamma_{ss}\times\gamma}\,ds
	=   \int_{\S} \IP{\gamma_{s}}{\gamma\times\gamma_{ss}}\,ds
	\,.
\]
Calculating from the cross product, we also have
\begin{align*}
	\gamma_{ss} &= \gamma_s\times(\gamma_s\times\gamma)
	+ \gamma\times(\gamma_{ss}\times\gamma + \gamma_s\times\gamma_s)
	\\&=
	               -\gamma
		       + \gamma\times(\gamma_{ss}\times\gamma)
		       \,.
\end{align*}
Now $\IP{\nu}{T} = 0$ and 
\[
	h_{11}\IP{\nu}{N} = \kappa + \IP{\gamma}{N} = \kappa - \kappa^{-1}
	\,,
\]
\[
	h_{11}\IP{\nu}{B} = \IP{\gamma}{B}
	\,.
\]
One interesting equation is
\[
	h_{11}^2 = \kappa^2 + 1 - 2(-1) = \kappa^2 + 3\,.
\]
\[
	\nu = \IP{\nu}{N}N + \IP{\nu}{B}B
\]
\[
	h_{11} = \kappa\IP{\nu}{N}
\]
\[
	\IP{\gamma}{B}_s = \IP{\gamma}{-\tau N} = -\tau\IP{\gamma}{N}
\]
\[
	(\IP{\gamma}{B}/\IP{\nu}{B})_s
	= -\tau\IP{\gamma}{N}/\IP{\nu}{B} + \tau\IP{\gamma}{B}\IP{\nu}{N}/\IP{\nu}{B}^2 - \IP{\gamma}{N}\IP{\nu_s}{B}/\IP{\nu}{B}^2
\]
\end{proof}

\begin{thm}
Suppose $F:M^2\rightarrow\R^3$ is a smooth starshaped CMC surface with free boundary on
$\S^2\subset\R^{3}$.
Then $F(M^2)$ is a union of umbilics, that is, pieces of planes and spheres. In
particular, if $M^2$ is connected, $F(M^2)$ is a flat disk or spherical cap.
\end{thm}
\begin{proof}
We may assume without loss of generality that $H \ge 0$. (If not, perform the
proof for the new surface $\hat F:M^2\rightarrow\R^3$ defined by reversing the
orientation of $F$ so that $\hat H = -H \ge 0$.)

Note that in the following calculations boundary integrands are always
evaluated in Fermi coordinates where the $n$-th direction is taken to be normal
to the boundary pointing away from the interior of $\S^n$.
Since
\[
\Delta |F|^2 = -2H\IP{F,\nu} + 2n
\]
we have
\[
2\int_{\partial M} \IP{F}{\nabla_n F}\,dS
 = -2H\int_M \IP{F}{\nu}\,d\mu + 2n\mu(M)
\,.
\]
Now $\Delta \IP{F}{\nu} = H - |A|^2\IP{F}{\nu}$ so
\[
\int_M H-|A|^2\IP{F}{\nu}\,d\mu
 = \int_{\partial M} h_{nn}\,dS
\,.
\]
Combining these yields
\begin{equation}
\int_{\partial M} (H - nh_{nn})\,dS
 = \int_M (n|A|^2 - H^2)\IP{F}{\nu}\,d\mu
 \ge 0
\label{EQumbilic}
\end{equation}
with equality iff $F$ is umbilic.

Now we specialise to $n=2$. We calculate
\[
\int_{\partial M} (H-nh_{22})\,dS
 = -\int_{\partial M} H\,dS +
  + 2\int_{\partial M} h_{11}\,dS
 = -HL(\partial M)
\,,
\]
by the two lemmata above.
Thus
\[
\int_{\partial M} (H-nh_{22})\,dS
 = -HL(\partial M)
 \le 0
\,.
\]
Combining this with \eqref{EQumbilic} yields the result.
\end{proof}

%A higher-dimensional version seems possible.
%
%\begin{thm}
%Suppose $F:M^n\rightarrow\R^{n+1}$ is a smooth starshaped CMC surface with free boundary on
%$\S^n\subset\R^{n+1}$.
%Then $F(M^n)$ is a union of umbilics, that is, pieces of planes and spheres. In
%particular, if $M^n$ is connected, $F(M^n)$ is a flat disk or spherical cap.
%\end{thm}
%\begin{proof}
%We may assume without loss of generality that $H \ge n$. (If not, perform the proof for the new surface $\hat F:M^2\rightarrow\R^3$ defined by reversing the orientation of $F$ so that $\hat H = -H \ge n$.)
%
%Note that in the following calculations boundary integrands are always
%evaluated in Fermi coordinates where the $n$-th direction is taken to be normal
%to the boundary pointing away from the interior of $\S^n$.
%Since
%\[
%\Delta |F|^2 = -2H\IP{F,\nu} + 2n
%\]
%we have
%\[
%2\int_{\partial M} \IP{F}{\nabla_n F}\,dS
% = -2H\int_M \IP{F}{\nu}\,d\mu + 2n\mu(M)
%\,.
%\]
%Now $\Delta \IP{F}{\nu} = H - |A|^2\IP{F}{\nu}$ so
%\[
%\int_M H-|A|^2\IP{F}{\nu}\,d\mu
% = \int_{\partial M} h_{nn}\,dS
%\,.
%\]
%Combining these yields
%\begin{equation}
%\int_{\partial M} (H - nh_{nn})\,dS
% = \int_M (n|A|^2 - H^2)\IP{F}{\nu}\,d\mu
% \ge 0
%\label{EQumbilic}
%\end{equation}
%with equality iff $F$ is umbilic.
%
%We calculate
%\[
%\int_{\partial M} (H-nh_{22})\,dS
% = -n\int_{\partial M} h_{22}\,dS + H\mu^\partial(\partial M)
%\,,
%\]
%%since the second fundamental form on the boundary is the identity, in particular constant, we have also in particular $H^\partial$ constant (and equal to $(n-1)$).
%Then, since $H$ is also constant, we conclude that on the boundary $h_{nn}$ must also be constant.
%This implies
%\[
%\int_{\partial M} (H-nh_{22})\,dS
% = -n(H-H^\partial)\mu^\partial(\partial M) + H\mu^\partial(\partial M)
% = (n-1)(n-H)L(\partial M)
% \le 0
%\,.
%\]
%Combining this with \eqref{EQumbilic} yields the result.
%\end{proof}

We establish some conventions and basic formulae. Let \(D\) denote the Euclidean connection. Let \(g\) denote metric on \(TM\) induced by the embedding \(M \to (\R^3, \IP{\cdot}{\cdot})\) and let \(\nabla\) be the corresponding Levi-Civita connection. Then for any \(X, Y \in T_xM\)
\begin{equation}
\label{eq:levi_civita}
\nabla_X Y = \pi_{TM} (D_X Y)
\end{equation}
where \(\pi_{TM}\) is the orthogonal projection, \(T\R^3 \to TM\) over \(M\) onto the tangent bundle \(TM\).

We define the second fundamental form, \(h\) of \(M\) by
\[
h(X, Y) = \pi_{\nu M} (D_X Y)
\]
where \(\pi_{\nu M}\) is the orthogonal projection \(T\R^3 \to \nu M\) over \(M\) onto the normal bundle \(\nu M\). Then we may write
\begin{equation}
\label{eq:gauss}
D_X Y = \nabla_X Y + h(X, Y) \nu^M
\end{equation}
so that
\begin{equation}
\label{eq:sff}
h(X, Y) = \IP{D_X Y}{\nu^M} = \IP{Y}{-D_X \nu^M}.
\end{equation}

Now we establish our conventions relating to \(\partial M\). First, along \(\partial M\) we have
\begin{equation}
\label{eq:boundary_orthgonal}
\nu^M \perp \nu^{\S^2} = F = \nu^{\partial M}
\end{equation}
where \(F\) is the position vector field in \(\R^3\), \(\nu^M\) is the unit normal to \(M \subseteq \R^3\), \(\nu^{\S^2}\) is the unit normal to \(\S^2 \subseteq \R^3\) and \(\nu_{\partial M}\) is the unit normal to \(\partial M \subseteq M\).

Let us write \(\gamma\) for a connected component of \(\partial M\) and parametrise it by arc-length. Let \(T = \gamma_s\) be the unit tangent vector. Write
\begin{equation}
\label{eq:euclidean_kappa}
D_T T = \gamma_{ss} = \kappa^{\R^3} N^{\R^3}
\end{equation}
with \(\kappa^{\R^3}\) the geodesic curvature of \(\gamma\) (defined up to sign) as a curve in \(\R^3\) and \(N^{\R^3}\) the unit normal to \(\gamma\) in \(\R^3\). Note that since \(\gamma \subseteq \S^2\), we have \(\kappa^{\R^3} \ne 0\) and so \(N^{\R^3} = \tfrac{\gamma_{ss}}{|\gamma_{ss}|}\) is well defined.

Let us also write
\begin{equation}
\label{eq:boundary_kappa_vector}
\vec{\kappa} = \pi_{TM} (\gamma_{ss}) = \gamma_{ss} - \IP{\gamma_{ss}}{\nu^M} \nu^M.
\end{equation}
That is the curvature vector, \(\vec{\kappa}\) in \(M\) is the projection of \(\gamma_{ss}\), the Euclidean curvature vector onto the tangent space to \(M\). Note \(\{T, \nu^{\partial M}\}\) forms an oriented, orthonormal basis of \(TM\) along \(\partial M\) so that \(N = \nu^{\partial M}\) is a unit length normal field to \(\gamma\)  in \(M\). Then as \(\vec{\kappa} \perp T\) we may write 
\begin{equation}
\label{eq:boundary_kappa}
\vec{\kappa} = -\kappa N, \quad \kappa = -\IP{\vec{\kappa}}{N} = -\IP{\gamma_{ss}}{\gamma}.
\end{equation}
Finally, using equation \eqref{eq:gauss} for the second fundamental form and equations \eqref{eq:euclidean_kappa}, \eqref{eq:boundary_kappa_vector} and \eqref{eq:boundary_kappa} for the geodesic curvatures we have 
\[
\kappa N = -\pi_{TM} (D_T T) = -\pi_{TM} (\nabla_T T + h(T, T) \nu^M) = -\nabla_T T.
\]
Since \(T \perp N\) we also have
\[
0 = \partial_T [g(T, N)] = g(\nabla_T T, N) + g(T, \nabla_T N) = -\kappa + g(T, \nabla_T N).
\]
As \(N\) is unit length, \(\nabla_T N \perp N\) and hence we obtain the Frenet-Serret equations,
\begin{equation}
\label{eq:frenet_serret}
\nabla_T T = -\kappa N, \quad \nabla_T N = \kappa T.
\end{equation}

We have a simple lemma regarding the curvature of \(M\) and \(\partial M\) along \(\partial M\).

\begin{lem}[{\cite[Lemma 2]{ros1995stability}}]
\label{lem:boundary_curvature}
Let \(M^2 \subseteq \R^3\) be a hypersurface with boundary \(\partial M \subseteq \S^2\) meeting \(\S^2\) orthogonally and let \(\gamma\) be a connected component of \(\partial M\) with unit tangent \(T\) and normal vector \(N\) in \(M\).

Then, \(\kappa \equiv 1\) and \(\{T, N\}\) digonalises \(h\) with respect to \(g\) along \(\gamma\).
\end{lem}

\begin{proof}
From \(|\gamma| = 1\), we have \(\IP{\gamma_s}{\gamma} = 0\) and hence
\[
0 = \partial_s \IP{\gamma_s}{\gamma} = \IP{\gamma_{ss}}{\gamma} + \IP{\gamma_s}{\gamma_s}.
\]
Thus, using definition \eqref{eq:boundary_kappa},
\[
\kappa = -\IP{\gamma_{ss}}{\gamma} = \IP{\gamma_s}{\gamma_s} = 1.
\]

Using the orthogonal boundary condition in equation \eqref{eq:boundary_orthgonal}: \(N = \nu^{\partial M} = F(\gamma) = \gamma\), we obtain
\[
\IP{D_T \nu^M}{N} = -\IP{\nu^M}{D_T N} = -\IP{\nu^M}{D_T \gamma} = -\IP{\nu^M}{T} = 0.
\]
That is, \(D_T \nu^M \perp N\) and since \(D_T \nu^M \in TM = \text{span} \{T, N\}\) we have \(D_T \nu^M\) is proportional to \(T\). The formula \eqref{eq:sff} for \(h\) is \(h(T, T) = -\IP{D_T \nu^M}{T}\) so that
\[
-D_T \nu^M = h(T, T) T.
\]
Likewise,
\[
\IP{D_{N} \nu^M}{T} = -\IP{\nu^M}{D_{N} T} = -\IP{\nu^M}{D_T N} = \IP{D_T \nu^M}{N} = 0,
\]
and hence
\[
-D_N \nu^M = h(N, N) N.
\]

Therefore \(\{T, N\}\) diagonalises \(h\) with respect to \(g\) along \(\gamma\) since again from the formula \eqref{eq:sff},
\[
h(X, Y) = g(-D_X \nu^M, Y) = g(-D_Y \nu^M, X).
\]
Note that we also have
\[
h(T, N) = - \IP{D_T \nu^M}{N} = 0.
\]
\end{proof}

\begin{rmk}
We now have using the definition of the second fundamental form, \eqref{eq:gauss} and Lemma \ref{lem:boundary_curvature},
\[
\gamma_{ss} = \kappa^{\R^3} N^{\R^3} = D_T T = \nabla_T T + h(T, T)\nu^M = -\kappa N + h(T, T) \nu^M = -N + h(T, T) \nu^M.
\]
In particular we may relate the various curvatures along \(\gamma\):
\begin{equation}
\label{eq:boundary_curvature_relations}
(\kappa^{\R^3})^2 = \kappa^2 + h(T, T)^2 = 1 + h(T, T)^2
\end{equation}
since \(N \in TM\) so that \(\{N, \nu^M\}\) are orthonormal.
\end{rmk}

Next we compute the variation of \(h(N, N)\) along \(\partial M\).

\begin{lem}
\label{lem:hNN_variation}
Let \(M^2 \subseteq \R^3\) be a hypersurface with boundary \(\partial M \subseteq \S^2\) meeting \(\S^2\) orthogonally and let \(\gamma\) be a connected component of \(\partial M\) with unit tangent \(T\) and normal vector \(N\) in \(M\).

Then,
\[
\partial_T [h(N, N)] = {\color{red} \text{what?}}
\]
\end{lem}

\begin{proof}
Parametrise a neighbourhood of \(\partial M\) in \(M\) by
\[
\varphi(r, s) = \exp^M_{\gamma(s)} (r N(\gamma(s))).
\]
Then
\[
\varphi_{\ast} \partial_r |_{r=0} = N, \quad \varphi_{\ast} \partial_s |_{r=0} T.
\]
Let us write
\[
\varphi_{\ast} \partial_r = \bar{N}, \quad \varphi_{\ast} \partial_s = \bar{T}
\]
so that \(\bar{T}, \bar{N}\) are extensions of \(T, N\) to the neighbourhood. That is, \(\bar{T}(r=0) = T\) and \(\bar{N}(r=0) = N\). We also have
\begin{equation}
\label{eq:commutator}
[\bar{T}, \bar{N}] = \varphi_{\ast} [\partial_s, \partial_r] = 0 \quad \Rightarrow \quad \nabla_{\bar{T}} \bar{N} = \nabla_{\bar{N}} \bar{T}.
\end{equation}
Since for each fixed \(s\), \(\mu_s : r \mapsto \exp^M_{\gamma(s)} (rN(\gamma(s)))\) is a geodesic with velocity vector \(\bar{N}\), we have
\begin{equation}
\label{eq:geodesic_variation}
\nabla_{\bar{N}} \bar{N} = 0.
\end{equation}

Let us also note that for each fixed \(s\), \(\mu_s (r) = \varphi(r, s)\) is a geodesic hence \(J_s (r) = \bar{T}(r, s)\) is a Jacobi field along \(\mu_s\) determined uniquely by
\begin{equation}
\label{eq:jacobi}
J_s(0) = T(s), \quad J_s'(0) = \nabla_{\bar{N}} \bar{T}|_{r=0} = \nabla_{\bar{T}} \bar{N}|_{r=0} = \nabla_T N = \kappa(s) T (s) = T (s)
\end{equation}
where we used equation \eqref{eq:commutator} to swap the order of differentiation and with the last equality following from Lemma \ref{lem:boundary_curvature}.

Now we have
\[
\partial_T [h(N, N)] = (\nabla_T h) (N, N) + 2h(N, \nabla_{T} N).
\]
But from the Frenet-Serret equation \eqref{eq:frenet_serret} and the fact that by Lemma \ref{lem:boundary_curvature}, \(\{T, N\}\) diagonalises \(h\) we get
\[
h(N, \nabla_T N) = h(N, \kappa T) = \kappa h(N, T) = 0.
\]
Then applying the Codazzi equation,
\[
\begin{split}
\partial_T [h(N, N)] &= (\nabla_T h) (N, N) = (\nabla_N h) (T, N) \\
&= \left[\partial_{\bar{N}} [h(\bar{T}, \bar{N})] - h(\nabla_{\bar{N}} \bar{T}, \bar{N}) - h(\bar{T}, \nabla_{\bar{N}} \bar{N})\right]|_{r=0}.
\end{split}
\]
The commutator equation \eqref{eq:commutator}, Frenet-Serret equation \eqref{eq:frenet_serret} and Lemma \ref{lem:boundary_curvature} again imply that
\[
h(\nabla_{\bar{N}} \bar{T}, \bar{N})|_{r=0} = h(\nabla_T N, N) = \kappa h(T, N) = 0,
\]
while the geodesic equation \eqref{eq:geodesic_variation} implies that
\[
h(\bar{T}, \nabla_{\bar{N}} \bar{N})|_{r=0} = 0.
\]

Therefore we obtain
\[
\partial_T [h(N, N)] = \partial_{\bar{N}} [h(\bar{T}, \bar{N})]|_{r=0}.
\]

{\color{red} Now what?}
\end{proof}

Thus we obtain the important corollary:
\begin{cor}
Let \(M^2 \subseteq \R^3\) be a \emph{CMC} hypersurface with boundary \(\partial M \subseteq \S^2\) meeting \(\S^2\) orthogonally.

Then \(\partial M\) is a disjoint union of circles of latitude, that is intersections of planes with \(\S^2\).
\end{cor}

\begin{proof}
We prove that each connected component, \(\gamma\) of \(\partial M\) is a circle of latitude. First, the fact that \(\partial M\) is CMC implies that
\[
h(T, T) + h(N, N) = H \equiv \text{ constant}.
\]
Therefore, by lemma \ref{lem:hNN_variation} we have
\[
\partial_T [h(T, T)] = - \partial_T [h(N, N)] = 0
\]
so that \(h(T, T)\) is constant along \(\partial M\).

Now by the remark following Lemma \ref{lem:boundary_curvature} we have
\[
(\kappa^{\R^3})^2 = 1 + h(T, T)^2 \equiv \text{ constant}.
\]

Thus \(\gamma\) is a constant curvature space curve that lies on the sphere.

{\color{red} Does this imply \(\gamma\) must be a circle in a plane?}
\end{proof}

\section*{acknowledgements}

\bibliographystyle{plain}
\bibliography{mbib}

\end{document}
