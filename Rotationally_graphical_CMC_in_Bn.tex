\documentclass[10pt]{amsart}

\usepackage{amssymb, amsthm, amsmath, graphicx, yfonts, wasysym, appendix, url, verbatim}
\usepackage[margin=4cm]{geometry}
\usepackage{rotating}
\usepackage{amsbsy,enumerate}
\usepackage{graphicx}
\usepackage{ccaption}
\usepackage{xcolor}
\usepackage{times}
\usepackage{wrapfig}
\usepackage{amsmath, amssymb, amsthm} %Also recommend the standard AMS LaTeX maths packages.

%%%%%%%%%%%%%%%%%%%%%%%%%%%%%%%%%%%%%%%%%
%%  Some useful macros.
\newcommand{\pD}[2]{\frac{\partial #1}{\partial #2}}
\newcommand{\IT}[1]{\intertext{\flushright\small\parbox{3in}{\flushright #1}}}
\newcommand{\HL}[1]{\textbf{\textit{#1}}}
\newcommand{\auth}[1]{\hskip+1in\textit{Authored by #1}.\vspace{1cm}}
\newcommand{\T}[1]{\tilde{#1}}
\newcommand{\myover}[2]{\genfrac{}{}{0pt}{}{#1}{#2}}
\newcommand{\vn}[1]{\lVert#1\rVert}
\newcommand{\IP}[2]{\left< #1 , #2 \right>}
\newcommand{\eIP}[2]{\left(\left. #1 \hskip+0.5mm\right| #2 \right)}
\newcommand{\rD}[2]{\frac{d #1}{d #2}}
\newcommand{\tpD}[2]{\text{\tiny$\frac{\partial #1}{\partial #2}$}}
\newcommand{\sfrac}[2]{\text{\fontsize{5}{5}\selectfont$\frac{#1}{#2}$}}

%%%%%%%%%%%%%%%%%%%%%%%%%%%%%%%%%%%%%%%%%
%%  Some non-LaTeX native symbols I use.
\newcommand{\arccosh}{\text{arccosh}}
\newcommand{\Q}{\ensuremath{\mathcal{Q{}}}}
\renewcommand{\P}{\ensuremath{\mathcal{P{}}}}
\newcommand{\R}{\ensuremath{\mathbb{R}}}
\newcommand{\D}{\ensuremath{\mathbb{D}}}
\newcommand{\N}{\ensuremath{\mathbb{N}}}
\renewcommand{\S}{\ensuremath{\mathbb{S}}}
\newcommand{\Z}{\ensuremath{\mathbb{Z}}}
\newcommand{\SD}{\ensuremath{\mathcal{D{}}}}
\newcommand{\SB}{\ensuremath{\mathcal{B{}}}}
\newcommand{\SW}{\ensuremath{\mathcal{W{}}}}
\newcommand{\SY}{\ensuremath{\mathcal{Y{}}}}
\newcommand{\SH}{\ensuremath{\mathcal{H{}}}}
\newcommand{\SK}{\ensuremath{\mathcal{K{}}}}
\newcommand{\SL}{\ensuremath{\mathcal{L{}}}}
\newcommand{\SA}{\ensuremath{\mathcal{A{}}}}
\newcommand{\SX}{\ensuremath{\mathcal{X{}}}}
\newcommand{\SI}{\ensuremath{\mathcal{I{}}}}
\newcommand{\SP}{\ensuremath{\mathcal{P{}}}}
\newcommand{\SU}{\ensuremath{\mathcal{U{}}}}
\newcommand{\SC}{\ensuremath{\mathcal{C{}}}}
\newcommand{\SG}{\ensuremath{\mathcal{G{}}}}
\newcommand{\SR}{\ensuremath{\mathcal{R{}}}}
\renewcommand{\SS}{\ensuremath{\mathcal{S{}}}}

\newcommand{\Div}{\text{div}}
\renewcommand{\L}{\ensuremath{\mathcal{L}}}
\newcommand{\g}{\textswab{g}}
\newcommand{\Cmss}{\ensuremath{C_{M\!S\!S}}}
\newcommand{\ts}[2]{\ensuremath{\text{$\textstyle\frac{#1}{#2}$}}}
\newcommand{\lb}{\ensuremath{<\hskip-2mm<}}
\newcommand{\rb}{\ensuremath{>\hskip-2mm>}}
\newcommand{\gmcfh}{GMCF${}^{H \le\, 0}$}

\allowdisplaybreaks[4]

%%%%%%%%%%%%%%%%%%%%%%%%%%%%%%%%%%%%%%%%%
%%  Theorem environments.
\newtheorem{thm}{Theorem}[section]
\newtheorem{cor}[thm]{Corollary}
\newtheorem{prop}[thm]{Proposition}
\newtheorem{lem}[thm]{Lemma}
\newtheorem{id}[thm]{Identity}
\newtheorem{est}[thm]{Estimate}
\newtheorem*{uthm}{Theorem}
\newtheorem*{uprop}{Proposition}
\newtheorem*{ulem}{Lemma}
\newtheorem*{defn}{Definition}
\newtheorem{conds}[thm]{Conditions}

\theoremstyle{remark}
\newtheorem*{rmk}{Remark}
\newtheorem*{note}{Note}
\newtheorem*{cond}{Condition}

\title{Rotationally Symmetric CMC in the Ball}
\author{}
\keywords{minimal hypersurface, mean curvature, free boundary, geometric
analysis} \subjclass[2000]{49Q05\and 53A10}


\begin{document}

\begin{abstract}
In this note we use the strong maximum principle to show that CMC hypersurfaces with free boundary in (for the outside we need a noncompact max principle working for free boundary hypersurfaces in the ball) the ball and graphical in a rotation of the Euclidean space are either flat or rotationally symmetric. This is independent of topology or dimension.
\end{abstract}

\maketitle

\section{Introduction}


\begin{thm}[Graphical in a rotation CMC are either flat or rotationally symmetric]
Let $F:M^n\rightarrow\R^{n+1}$ be a smooth rotation Killing-graphical CMC
hypersurface with free boundary on $\S^n\subset\R^{n+1}$. Let the graph direction $K$ be a rotation of the Euclidean space. Then $M^n = \D^n$
and $M := F(M^n)$ is a standard flat disk or it is rotationally symmetric with rotation $K$.
\label{thmuniqueness}
\end{thm}

%{{{
In the above statement, we use Killing-graphical to mean that the function
$s:M\rightarrow\R$ given by
\begin{equation}
\label{EQsV}
s(x) := \IP{\nu^M(x)}{V(x)}\,,
\end{equation}
where $\nu^M:M\rightarrow\R^{n+1}$ is a unit normal vector field along $F(M^n)$,
and $V:\R^{n+1}\rightarrow\R^{n+1}$ is a Killing field,
is positive.

\rmk{Weakly graphical only} We do not require the strict graphicallity. If that is true then we obtain uniqueness of CMC disks.
\section{Setting}%{{{

Consider the standard unit sphere in Euclidean space $\S^n = \{x\in\R^{n+1}\,:\,|x|=1\}\subset
\R^{n+1}$.
We use $\nu^{\S^n}:\S^{n}\rightarrow\R^{n+1}$ to denote its outer
normal vectorfield.
Let $M^n$ be a smooth, orientable $n$-dimensional Hausdorff paracompact manifold with boundary $\partial M^n$.
Let $g$ be a Riemannian metric on $M^n$.
Set $M:=F(M^n)\subset{\R}^{n+1}$ where $F:M^n\rightarrow {\R}^{n+1}$ is a
smooth isometric immersion satisfying
\begin{align}
&{\partial} M \ \equiv \ F({\partial} M^n)\ =\ M \cap \S^n,\notag\\
&\IP{{\nu}^M}{{\nu}^{\S^n}}(F(p))\ =\ 0, \text{ }\forall
p~\in{\partial} M^n.
\label{initial}
\end{align}
Since $F$ is isometric, the Riemannian structure induced by the embedding $F$
is the same as that given by $g$, that is $(M^n, g) = (M^n, F^*\delta)$, where
$\delta$ is the standard metric on $\R^n$.

Let us denote by $A:T M\times T M \rightarrow \R$ the second fundamental form
of $M$ with components given by $h_{ij}$ where $1\leq i,j\leq n$ where $h_{ij}=
A(\tau_i,\tau_j)$ for two sections $\tau_i$ and $\tau_j$ in $TM$.
For $\S^n$ we have $A^{\S^n}:T \S^n \times T\S^n\rightarrow \R$ the
second fundamental form with components $h^{\S^n}_{ij}$ for $1\leq i,j\leq
n$.
%}}}

\section{Auxiliary Equations}

%{{{
Let us define the quantities we use in the proof of the uniqueness theorem.
Recall the function $s:M\rightarrow\R$ defined in \eqref{EQsV} above.
When $V$ is a translation, following \cite{ecker1989mce} we term $s$ the \emph{graph quantity}.
If, up to reparametrisation,
\[
F(p) = (p,u(p)) = p_ie_i + u(p)V\,,
\]
then one can relate the gradient of the associated scalar function $u$ to the
reciprocal of the graph quantity $s$. This implies that a lower bound on $s$
is equivalent to a gradient bound for $u$.
Throughout this section we assume that $F:M^n\rightarrow\R^{n+1}$ is a smooth
minimal immersed hypersurface.

\begin{lem} The quantity $s:M\rightarrow\R$
satisfies
\begin{align*}
{\Delta}^M s = - |A|^2 s\,.
\end{align*}
where $\Delta^M$ is the Laplace-Beltrami operator on $M$.
\label{sevolution}
\end{lem}

Note that for any choice of orthonormal basis of the tangent space of $\S^n$
the second fundamental form of $\S^n$ is diagonal.
It was shown by Stahl \cite{stahl1996convergence} that on the boundary we have
\begin{align*}
h_{in}=h^{\S^n}_{in}=0\quad \text{ for all }\quad i=1,\ldots,n-1.
\end{align*}
Since $n-1$ of the tangent vectors are on the closed submanifold $\partial M\subset\S^n$ we can choose
our basis above such that on the boundary
\begin{align*}
h_{ij}=0\quad  \text{ if } \quad i\neq j\,,\quad \text{ and }\quad h^{\S^n}_{ij}=\delta_{ij}\quad\text{ if }\quad i\neq j.
\end{align*}
\section{Minimal graphical hypersurfaces with free boundary are flat disks}
\label{KillingSection}

Now assume that
\begin{align}
	s \geq 0 \label{graphcondition}\,.
\end{align}
We shall prove Theorem \ref{thmuniqueness}.

\begin{proof}
Apply the elliptic maximum principle to the evolution of $s0\geq 0$ to see that
there is no interior minima. On the boundary we note that $\nabla_{\nu_\S^n}s=0$ using
$=\nu^{\S^n}$, the Neumann boundary condition and the  of the second fundamental form on the boundary. 

\begin{align*}
\nabla_{\nu_\S^n}s_v = h_{nn}\IP{\nu^{\S^n}}{K} = 0\,.
\end{align*}
This implies that $s_v$ has no boundary minima either.  Therefore $s_v$ is constant everywhere on $\bar M$.

Returning to the evolution of $s_v$ on the interior this tells us that
\begin{align*}
{\Delta}^M s_V = - |A|^2 s_V\,,
\end{align*}
implying that either $M$ is a flat disc or $K\in TM$ everywhere on $M$, making $M$ rotationally symmetric. 
\end{proof}

\begin{rmk}
The bounded curvature is required for the evolution of the $s_v$.
\end{rmk}


\section*{acknowledgements}

\bibliographystyle{plain}
\bibliography{mbib}

\end{document}
